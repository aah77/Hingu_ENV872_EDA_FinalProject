% Options for packages loaded elsewhere
\PassOptionsToPackage{unicode}{hyperref}
\PassOptionsToPackage{hyphens}{url}
%
\documentclass[
  12pt,
]{article}
\title{Low Live Birth Weight Prevalence and Socioeconomic Factors in
Chicago}
\usepackage{etoolbox}
\makeatletter
\providecommand{\subtitle}[1]{% add subtitle to \maketitle
  \apptocmd{\@title}{\par {\large #1 \par}}{}{}
}
\makeatother
\subtitle{\url{https://github.com/aah77/Hingu_ENV872_EDA_FinalProject.git}}
\author{Aman Hingu}
\date{}

\usepackage{amsmath,amssymb}
\usepackage{lmodern}
\usepackage{iftex}
\ifPDFTeX
  \usepackage[T1]{fontenc}
  \usepackage[utf8]{inputenc}
  \usepackage{textcomp} % provide euro and other symbols
\else % if luatex or xetex
  \usepackage{unicode-math}
  \defaultfontfeatures{Scale=MatchLowercase}
  \defaultfontfeatures[\rmfamily]{Ligatures=TeX,Scale=1}
  \setmainfont[]{Times New Roman}
\fi
% Use upquote if available, for straight quotes in verbatim environments
\IfFileExists{upquote.sty}{\usepackage{upquote}}{}
\IfFileExists{microtype.sty}{% use microtype if available
  \usepackage[]{microtype}
  \UseMicrotypeSet[protrusion]{basicmath} % disable protrusion for tt fonts
}{}
\makeatletter
\@ifundefined{KOMAClassName}{% if non-KOMA class
  \IfFileExists{parskip.sty}{%
    \usepackage{parskip}
  }{% else
    \setlength{\parindent}{0pt}
    \setlength{\parskip}{6pt plus 2pt minus 1pt}}
}{% if KOMA class
  \KOMAoptions{parskip=half}}
\makeatother
\usepackage{xcolor}
\IfFileExists{xurl.sty}{\usepackage{xurl}}{} % add URL line breaks if available
\IfFileExists{bookmark.sty}{\usepackage{bookmark}}{\usepackage{hyperref}}
\hypersetup{
  pdftitle={Low Live Birth Weight Prevalence and Socioeconomic Factors in Chicago},
  pdfauthor={Aman Hingu},
  hidelinks,
  pdfcreator={LaTeX via pandoc}}
\urlstyle{same} % disable monospaced font for URLs
\usepackage[margin=2.54cm]{geometry}
\usepackage{graphicx}
\makeatletter
\def\maxwidth{\ifdim\Gin@nat@width>\linewidth\linewidth\else\Gin@nat@width\fi}
\def\maxheight{\ifdim\Gin@nat@height>\textheight\textheight\else\Gin@nat@height\fi}
\makeatother
% Scale images if necessary, so that they will not overflow the page
% margins by default, and it is still possible to overwrite the defaults
% using explicit options in \includegraphics[width, height, ...]{}
\setkeys{Gin}{width=\maxwidth,height=\maxheight,keepaspectratio}
% Set default figure placement to htbp
\makeatletter
\def\fps@figure{htbp}
\makeatother
\setlength{\emergencystretch}{3em} % prevent overfull lines
\providecommand{\tightlist}{%
  \setlength{\itemsep}{0pt}\setlength{\parskip}{0pt}}
\setcounter{secnumdepth}{5}
\usepackage{fvextra} \DefineVerbatimEnvironment{Highlighting}{Verbatim}{breaklines,commandchars=\\\{\}}
\ifLuaTeX
  \usepackage{selnolig}  % disable illegal ligatures
\fi

\begin{document}
\maketitle

\newpage
\tableofcontents 
\newpage
\listoftables 
\newpage
\listoffigures 
\newpage

\hypertarget{rationale-and-research-questions}{%
\section{Rationale and Research
Questions}\label{rationale-and-research-questions}}

Low Birth Weight refers to when a baby weighs less than 5 pounds 8
ounces (approximately 2.5 kg) at the time of birth. This definition was
adopted as a major benchmark for pregnancy success, as well as a
predictor for certain pregnancy complications (Taffel, 1975). For the
child, complications associated with low birth weight include
Respiratory Distress Syndrome (RDS), Intraventricular Hemmorhages, and a
weakened immune system (March of Dimes, 2021). A compromised immune
system places the child in a vulnerable state, drastically reducing
their ability to fight off infections. Approximately eight percent of
all babies born in the United States are born with low birth weight, but
this group constitutes over 75\% of all neonatal deaths in the United
States (Taffel, 1975).

Occurrences of low live birth weight vary widely based on physical
characteristics of the mother, pregnancy history, age, and marital
status. Many national public health organizations recognize these
factors, as well as other factors that pertain to the mother's
socioeconomic status. Relevant socioeconomic factors can include
household income level, proximity to high traffic volume areas, food
insecurity, and educational status. Factors such as income and food
insecurity are often intrinsically tied and deter natural maternal
weight gain, thereby increasing the likelihood of a low live birth
weight (Gourlay, 2021).

The connections between socioeconomic factors and maternal physical
health motivates the central research question of the project below:

\begin{enumerate}
\def\labelenumi{\arabic{enumi}.}
\tightlist
\item
  Do socioeconomic factors such as unemployment, educational attainment,
  and portion below poverty level explain variability in the prevalence
  of low birth weight?
\end{enumerate}

It is hypothesized that the amalgamation of all three factors will
produce the most statistically robust model in comparison to the
influence of any of the models alone.

\newpage

\hypertarget{dataset-information}{%
\section{Dataset Information}\label{dataset-information}}

The dataset to be used in this study is the Chicago Health Indicators
dataset, originally found on a website called Geo Data and Lab. The Geo
Data and Lab website is a repository of many different socioeconomic,
health, and environmental datasets. The data was originally provided by
the Illinois Department of Public Health and the US Census Bureau.This
dataset includes public health data for 77 community areas within the
greater Chicago Metropolitan Area, and was aggregated from 2005-2011.

Once attaining the dataset, an initial analysis was completed by running
head(), dim(), and summary() functions on the variables within the
dataset. After looking through the variables provided in the dataset, a
brief and informal literature review was conducted in order to deduce
which socioeconomic variables might bear the greatest level of influence
on low live birth weight. Parameters that were frequently discussed in
the literature included household income, level of education held by the
mother, and whether the mother (or other household members) possessed a
stable source of income.

Based on these commonly occurring parameters, the dataset was wrangled
to omit occurrences of NA, and selected for columns representing
``Portion under Poverty Level'', ``Unemployment Rate'', ``Community
Area'', '' ``Low Birth Weight Prevalence'', and ``Educational
Attainment''. A key characteristic in the behavior of socioeconomic
variables is their tendency to vary spatially. This phenomena is
especially true in urban metropolitan areas where costs of living and
income status are major factors in a family's choice of deciding where
to live. The area in which a family chooses to reside affects the ease
of access they may or may not have to nutrient dense food vendors (such
as supermarkets), as opposed to convenience stores. Factors such as
these may play larger roles in a mother's likelihood of having a child
with a low birth weight given the physiological relationship between a
mother's food intake and their child's physical health (Christian,
2020).

\begin{table}[!h]

\caption{\label{tab:Table 1}Variable Descriptions}
\centering
\resizebox{\linewidth}{!}{
\begin{tabular}[t]{l|l}
\hline
Variable & Description\\
\hline
Low Live Birth Weight & Percentage of all births in a community area with a birth weight less than 5.5 pounds.\\
\hline
Unemployment Rate & Unemployment as a percentage of persons aged 16 older.\\
\hline
Educational Attainment & Percentage of persons aged 25 or older without a high school diploma.\\
\hline
Portion under Poverty Level & Percentage of households in a community area living under poverty level.\\
\hline
\end{tabular}}
\end{table}

\newpage

\hypertarget{exploratory-analysis}{%
\section{Exploratory Analysis}\label{exploratory-analysis}}

To further confirm whether the socioeconomic variables chosen for in
this dataset vary spatially, spatial heat maps of the dependent and
independent variables chosen were created and are displayed below using
the initial spatial dataframe.

\begin{figure}
\centering
\includegraphics{Hingu_ENV872_Project_files/figure-latex/Spatial Analysis 1-1.pdf}
\caption{The figure depicts a spatial distribution map of Low Birth
Weight Prevalence across the Chicago Metropolitan Area}
\end{figure}

\begin{figure}
\centering
\includegraphics{Hingu_ENV872_Project_files/figure-latex/Spatial Analysis 2-1.pdf}
\caption{The figure depicts a spatial distribution of unemployment rates
across the Chicago Metropolitan Area}
\end{figure}

\begin{figure}
\centering
\includegraphics{Hingu_ENV872_Project_files/figure-latex/Spatial Analysis 3-1.pdf}
\caption{The figure depicts a spatial distribution of portion of
population under poverty level across the Chicago Metropolitan Area}
\end{figure}

\begin{figure}
\centering
\includegraphics{Hingu_ENV872_Project_files/figure-latex/Spatial Analysis 4-1.pdf}
\caption{The figure depicts a spatial distribution of educational
attainment across the Chicago Metropolitan Area}
\end{figure}

Observation of the spatial plots above indicates disparities and an
obvious variability in the distribution of these socioeconomic
variables. For the spatial plot of low live birth weight prevalence,
there is more pronounced density in the central eastern and southwest
community areas of the Chicago Metropolitan area. The next section
proceeds with linear regression analyses to quantitatively explore the
influence of the socioeconomic variables of interest on low live birth
weight prevalence.

\newpage

\hypertarget{analysis}{%
\section{Analysis}\label{analysis}}

\hypertarget{question-do-socioeconomic-factors-such-as-unemployment-educational-attainment-and-portion-below-poverty-level-explain-variability-in-the-prevalence-of-low-birth-weight}{%
\subsection{Question: Do socioeconomic factors such as unemployment,
educational attainment, and portion below poverty level explain
variability in the prevalence of low birth
weight?}\label{question-do-socioeconomic-factors-such-as-unemployment-educational-attainment-and-portion-below-poverty-level-explain-variability-in-the-prevalence-of-low-birth-weight}}

The overall analysis goal was to create a linear regression model with a
significant p-value that optimizes the adjusted R-squared value. Thus,
the process here was to first create a correlation plot to get an idea
of which of the variables of interest had the strongest correlation
value to low live birth weight and then proceed with creating a model
from that. Next step in the process included creating multiple linear
regression models, visualizing their model fit through residual plots \&
QQ plots, and looking at their coefficients, p-values, and R-squared
values. The last step of the process included creating another model to
see if a higher R-squared value could be achieved while maintaining
significance.

Unemployment and Portion under Poverty Level had the two highest
correlation values, and thus my first multiple linear regression model
consisted of looking at those 2 variables as predictors of low live
birth weight prevalence. When visualizing the model fit, the QQ plot
yielded most of the points on the 1/1 line, but there was assymetry in
the residuals plot indicating poor model fit. The p-value was
significant for the model, and the adjusted R-squared value was 0.532,
meaning that just about half of the variability in low live birth weight
can be explained by unemployment and poverty level.

Given the lower R-squared value, another multiple linear regression
model was created that included the third variable, Education Level,
that was originally left out. This model had a very similar poor model
fit as evidenced by the skewed residuals vs.~fitted plot. The model was
also significant, but had a slightly higher adjusted R-squared value.
The results of both models are organized into Table 2.

\begin{table}[!h]

\caption{\label{tab:unnamed-chunk-1}Model Results}
\centering
\resizebox{\linewidth}{!}{
\begin{tabular}[t]{r|l|l|r}
\hline
Model & Variables & P & R\\
\hline
1 & Unemployment, Portion Under Poverty Level & ***2.35E-13 & 0.5320\\
\hline
2 & Unemployment, Portion Under Poverty Level, Educational Attainment & ***6.36E-15 & 0.5941\\
\hline
\end{tabular}}
\end{table}

\newpage

\hypertarget{summary-and-conclusions}{%
\section{Summary and Conclusions}\label{summary-and-conclusions}}

The model that was both significant and resulted in the highest adjusted
R-squared value was Model 2. Model 2, which included all three
independent variables of interest, resulted in an R-squared value of
0.5941. This means that approximately 59\% of the variability in Low
Birth Weight Prevalence can be explained by Unemployment Rate,
Educational Attainment, and Portion Under Poverty Level. While this was
a greater fit in comparison to Model 1, Model 2 is still relatively weak
and could be improved.

For further improvements, researchers should consider exploring other
variables which are known to have some level of correlation to low birth
weight from the literature. One example of such a variable is air
quality (Sarizadeh, 2020). This study indicated that, due to
physiological changes in a mother's body, mothers were particularly more
sensitive to higher concentrations of common air pollutants. Common air
pollutants included in this study were Ozone, PM 10, and PM 2.5. Air
quality can also vary regionally depending on a particular region's
proximity to industrial activity, such as oil and gas plants, paper
mills, etc.

Researchers can also consider variables commonly associated with
malnutrition in mothers, which is a leading cause of low live birth
weight (Sarizadeh, 2020). These variables can include proximity towards
supermarkets, or even the concentration of supermarkets within a
particular community area. Food insecurity associated with residing in a
food desert is known to be associated with malnutrition (Christian,
2020).

Including these variables in a multivariable linear regression may lead
to a model in which more of the variability in low live birth weight
prevalence is accounted for. Local public health agencies can utilize
models such as these to drive policy based changes to support mothers at
a higher risk of having a child with a low birth weight by virtue of
their socioeconomic status.

\newpage

\hypertarget{references}{%
\section{References}\label{references}}

Christian, Vikram J., et al.~``Food Insecurity, Malnutrition, and the
Microbiome - Current Nutrition Reports.'' SpringerLink, Springer US, 10
Nov.~2020,
\url{https://link.springer.com/article/10.1007/s13668-020-00342-0}.

Sarizadeh, Reihaneh et al.~``The Association Between Air Pollution and
Low Birth Weight and Preterm Labor in Ahvaz, Iran.'' International
journal of women's health vol.~12 313-325. 4 May. 2020,
\url{doi:10.2147/IJWH.S227049}

March of Dimes. Low Birthweight,
\url{https://www.marchofdimes.org/complications/low-birthweight.aspx\#}.

Coleman-Jensen A, Gregory C, Singh A. Household food security in the
United States in 2013. 2014.
\url{https://www.ers.usda.gov/webdocs/publications/45265/48787_err173.pdf}.
Accessed 5 Nov 2020.

Hanson KL, Connor LM. Food insecurity and dietary quality in US adults
and children: a systematic review. Am J Clin Nutr. 2014;100(2):684--92.
\url{https://doi.org/10.3945/ajcn.114.084525}. Comprehensive systematic
review on dietary quality in food secure households.

Ritz B, Wilhelm M, Hoggatt KJ, Ghosh JK. Ambient air pollution and
preterm birth in the environment and pregnancy outcomes study at the
University of California, Los Angeles. Am J Epidemiol.
2007;166(9):1045--1052. \url{doi:10.1093/aje/kwm181}

Giovannini N, Schwartz L, Cipriani S, et al.~Particulate matter (PM10)
exposure, birth and fetal-placental weight and umbilical arterial pH:
results from a prospective study. J Matern Fetal Neonatal Med.
2018;31(5):651--655. \url{doi:10.1080/14767058.2017.1293032}

\end{document}
